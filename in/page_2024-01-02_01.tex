\section{Worldbuilding Maxims}
\emph{Chronological ID:} \texttt{2024-01-02:01}

\emph{Structural ID:} \texttt{1.1}

The author has pursued creative projects similar in the ``intended feel'' to the Madelineverse since 2012, and has used the name ``Madelineverse'' itself since 2021. Though these projects, in general\footnote{That is, with the exception of \emph{Post-SCrash Session 3: Spectators of the Host}.}, are considered to be overall failures, the lessons that they taught the author with regard to what works and does not work for the current iteration of the Madelineverse are considered invaluable, and have led to certain creative processes that might not be shared by others with a similar passion. An outline of these creative processes is presented here.

Through inspiration from worldbuilding-focused YouTube channels like Artifexian, the \textsc{physical possibility} of a setting is the first and foremost priority. In building a solar system, this would mean adherence to the laws of orbital mechanics, as well as tendencies observed in the composition of planets. In building a language, this would mean adherence to beliefs about what is naturalistic in a language, as well as tendencies observed as languages evolve, et cetera.

The worldbuilding errs on the side of ``\textsc{too human}''; while some of the more alien features of biology and languages might actually be more common throughout the universe, this is not considered for the purposes of Madelineverse worldbuilding. This goes back to the desire expressed in section \texttt{1}: to have the setting be conducive to a fictional story that, in the end, humans could appreciate in its own right. However, certain large-scale descisions have been made that set the Madelineverse's world apart from the Earth.

Conlanging is pursued using the usual method of \textsc{simulating evolution} from a proto-language. However, due to the author's current inexperience in conlanging, artificiality in the proto-language is expected; for example, there might be ``too few'' forms of certain common words and they might be ``too regular'' for a natural language. That being said, this, too, is seen as a challenge: to have the artificiality of any language created earlier in the process fit the greater worldbuilding.
\newpage

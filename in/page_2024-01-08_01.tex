\section{Geophysical Introduction}
\emph{Chronological ID:} \texttt{2024-01-08:01}

\emph{Structural ID:} \texttt{2}

Our attention is turned towards a star system with a single G4.9V-class star. The star is roughly 6 billion Earth years old, and is expected to remain in the main sequence for another 5.5 billion Earth years. It is orbited by a single rocky planet, appearing in the star's habitable zone (from 0.86 to 1.24 AU), and six gas giants outside the rocky planet's orbit. The gas giants have thoroughly cleared interplanetary space of any debris, and therefore, there isn't an asteroid belt to speak of, though lone asteroids may still be observed.

The four innermost gas giants are in a 1:2:4:8 orbital resonance with each other. It is thought to have formed naturally, due to the gas giants being more massive compared to those in other star systems around the galaxy. The outermost two gas giants are not in a resonance with the other four, but their orbits are also trending towards an eventual orbital resonance in the future. As is usual with gas giants, especially those as massive as ones in this system, each of the six gas giants has an assortment of major moons orbiting it, though the largest is no bigger than 7,000 km in diameter.

The rocky planet is also an unusual one within the galaxy, for two reasons. For one, it is a binary planet with a tight orbit, no larger than 60,000 km in semi-major axis. For another thing, both components of the binary planet have a larger-than-usual proportion of iron in their cores, allowing for Earth-like gravity on the larger member of the binary, despite its comparatively smaller diameter. However, only the larger member of the binary is massive enough to form a dense atmosphere to support complex multicellular life.

However, even taking multicellular life as a given, there are at least a couple of hurdles that it must overcome before it can achieve interplanetary, let alone interstellar, travel. Many of them are unique to the society that the planet hosts, that, for example, strangely prefers electric transportation over chemical equivalents, not allowing for chemical rockets to ever be developed. However, at least one issue is due to the planet's status as a binary: satellites in the planet's orbit will inevitably be perturbed by the larger-than-average moon and either return to the planet's atmosphere or be ejected out of the system entirely.
\newpage

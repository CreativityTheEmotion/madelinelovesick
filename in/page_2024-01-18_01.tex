\section{Presentation of Numerical Data}
\emph{Chronological ID:} \texttt{2024-01-18:01}

\emph{Structural ID:} \texttt{1.4}

Certain aspects of worldbuilding, especially the description of various celestial bodies, require presentation of precise numerical data. While it does not matter that the presented data includes margins of error, as it is for the convenience of the worldbuilder and not an in-universe description derived by in-universe scientists, there are still two major concerns.

The first of these concerns is that the data does not contradict itself; for example, given a spherical body's radius, calculation of its volume is trivial, and if the body's mass is also known, that allows calculation of many more of its attributes, such as density, surface gravity, escape velocity and so on. Whenever only one object, such as a planet, is described, these derivative values are calculated and presented; however, if a single page strives to describe many objects, such as a gas giant's moons, and physical space is limited, the choice has to be made to limit the description to values with no trivial connections.

The second of these concerns is that the data, especially concerning the more important bodies such as planets, is sufficient enough to derive observations across a wide variety of fields. As an example, ``albedo'' can typically mean either the geometric or Bond albedo, and both values are useful: the geometric albedo can be used to calculate an object's apparent magnitude, while the Bond albedo can be used to calculate an object's mean surface temperature, provided details about its atmosphere are also accounted for. The two are not interchangeable, neither can be trivially derived from the other, and when ``building'' a body, the value of both must be provided.

Lastly, this page will touch upon the temporary numerical designations assigned to objects, which, though this might not be intuitive, are also numbers describing them. Taking a gas giant's moons as an example, the first to receive numbers are the major moons, in order of their orbits (closest to furthest), and then, the minor moons, in order of their mass (heaviest to lightest). Similar logic applies to the planets excluding the worldbuilding project's ``home world'', et cetera.
\newpage

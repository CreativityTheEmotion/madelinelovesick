\section{The Presence of the *Moon in the *Earth's Ecosystem and Culture}
\emph{Chronological ID:} \texttt{2024-01-19:01}

\emph{Structural ID:} \texttt{2.2.1.1}

The mutual tidal locking of the *Earth, the only planet to harbor complex life in its planetary system, and the *Moon, its only major moon, by itself, leads to several important considerations in various fields, including geology, evolution of the pre-intelligent species and the culture of the intelligent species.

\begin{itemize}
  \item During the early history of the *Earth and the *Moon, there were tremendous amounts of tidal heating; at one point, this perfectly offset the lower luminosity of the *Sun, resulting in conditions on the *Earth similar to the modern day.
  \item As the mutual tidal locking set in, tidal acceleration and deceleration stopped being a concern; there is only orbital decay due to exchange of gravitational energy, occurring over timespans significantly longer than the host star's lifetime.
  \item The near sides of the *Earth and the *Moon attract each other more strongly, resulting in higher altitude variation, higher sea level, thicker atmosphere and greater tidal heating when compared to the far sides.
  \item The constant position of the *Moon in the *Earth's sky, not affected by libration, allows for easy and precise global reckoning, perhaps even for pre-intelligent species. This, therefore, is likely to be remarked by the intelligent species, which develops two ``latitudes'' to track locations on the *Earth: that radiating away from the poles (traditional latitude) and that radiating away from the near point (to be given a name in the future).
  \item The notion of a lunar, solar and lunisolar calendar is meaningless to the intelligent species of the *Earth, as the month (the *Moon's orbital period) and the day (the *Earth's rotational period) are of the same length. However, as mentioned in the pages about the gas giants, there is still a third, distinct length of time from the year and the month, the ``cosmic year'' associated with the innermost four gas giants.
\end{itemize}
\newpage

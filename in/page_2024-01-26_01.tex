\section{The Minor Planets of the Planetary System}
\emph{Chronological ID:} \texttt{2024-01-26:01}

\emph{Structural ID:} \texttt{2.9}

The minor planets orbiting the *Sun can be broadly categorised in a few different ways. The first and most obvious way is to divide them based on their semi-major axis, which leads to the establishment of an inner debris disc and an outer debris disc.

The inner debris disc, early in the planetary system's history, used to contain a few large planetesimals, some of which may have even qualified as planets. However, they were largely knocked out of their orbits to collide with either the *Sun or *GG1 by a planet in a high-eccentricity orbit, which then collided with an extrasolar object to create the *Earth and the *Moon, in a low-eccentricity orbit. The remaining inner debris disc objects are generally very small, and none of them likely fit the IAU definition of dwarf planet.

The outer debris disc has a more straightforward history, as remnants of the protoplanetary disc that did not coalesce to a planet. It can further be subdivided into objects that have resonant orbits with *GG6, of which those having a 2:3 resonance (for a semi-major axis of 129.50 AU and an orbital period of 1875 Earth years) and those having a 1:2 resonance (for a semi-major axis of 181.11 AU and an orbital period of 2500 Earth years) are the most common.

Minor planets in the planetary system can also be categorised by size, much like the moons of the seven planets. The convention used in this treatise is to divide them by mass, so that an object with a mass over 10\textsuperscript{24} kg is considered a planet, an object with a mass between 10\textsuperscript{22} and 10\textsuperscript{24} kg is considered major, an object with a mass between 10\textsuperscript{19} and 10\textsuperscript{22} kg is considered middling and an object with a mass below 10\textsuperscript{19} kg is considered minor.

One last way to categorise minor planets in the planetary system is by spectral type, which gives some indication as to what material they're made of. The two most prominent spectral type of minor planets, both in the inner and the outer debris discs, are C-type, primarily composed of carbon compounds, and S-type, primarily composed of silicon compounds.
\newpage

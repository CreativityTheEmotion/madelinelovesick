\section{A Proposed History for the Planetary System}
\emph{Chronological ID:} \texttt{2024-01-27:01}

\emph{Structural ID:} \texttt{2.10}

ca. 6,018 Mya, the planetary system formed from a protoplanetary disc and the star within started undergoing fusion of hydrogen into helium. In the inner planetary system, defined as the region lying within the star's frost line approximately 4.4 AU away from the star, two rocky planets formed, and in the outer planetary system, outside the frost line, six gas giants formed. The innermost gas giant was particularly large and could perturb orbits rather far away, affecting both the outer planetary system, in which the innermost four gas giants attained resonant orbits, and the inner planetary system.

ca. 5,880 Mya, the two rocky planets collided, forming a single moon-less planet on a high-eccentricity, low-inclination orbit, with an aphelion of around 2 AU and a perihelion of around 0.7 AU. Eventually, together with the innermost gas giant, it cleared its orbit of most asteroids in the inner debris disc.

ca. 5,017 Mya, a rogue extrasolar rocky planet was captured by the planetary system due to interaction with the innermost gas giant. Its orbit was unstable, and due to additional interaction with the gas giants, its perihelion ended up within the outer reaches of the star's atmosphere.

This caused the rogue planet to start losing orbital energy. However, before it could be swallowed by the star entirely, it collided with the remaining rocky planet. The resultant debris created two planetesimals in a low-eccentricity, high-inclination orbit, which became mutually tidally locked, and various small debris, which would collide with the binary planet again over geological timescales.

The rogue planet was an almost pure iron ball, thought to be either debris of a neutron star collision or a gas giant stripped of its outer layers. After impact, the components of the binary planet formed large liquid iron cores, ensuring a very strong magnetic field.

Roughly 3,000 Mya, life formed on the larger member of the binary planet via abiogenesis.
\newpage

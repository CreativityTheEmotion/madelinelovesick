\section{Standard for Categorising Objects for Celestial Fact Sheets}
\emph{Chronological ID:} \texttt{2024-01-29:01}

\emph{Structural ID:} \texttt{1.4.1}

For the purposes of this and other ``standards'' describing the description of celestial objects\footnote{Meta-descriptions, if you will.}, the first task that one needs to undertake is classify the object based on mass. The possible categories are:

\begin{itemize}
  \item \emph{M} $>$ 10\textsuperscript{29} kg: stellar object (similar to the Sun)
  \item 10\textsuperscript{25} kg $<$ \emph{M} $<$ 10\textsuperscript{29} kg: gas giant (similar to Jupiter)
  \item 10\textsuperscript{24} kg $<$ \emph{M} $<$ 10\textsuperscript{25} kg: terrestrial planet (similar to the Earth)
  \item 10\textsuperscript{22} kg $<$ \emph{M} $<$ 10\textsuperscript{24} kg: major sub-planetary object (similar to the Moon)
  \item 10\textsuperscript{19} kg $<$ \emph{M} $<$ 10\textsuperscript{22} kg: middling sub-planetary object (similar to 1~Ceres)
  \item \emph{M} $<$ 10\textsuperscript{19} kg: minor sub-planetary object (similar to 243~Ida)
\end{itemize}

The reasons why these specific limits were picked are as follows:

\begin{itemize}
  \item 10\textsuperscript{29} kg falls within the range at which an object is considered a ``brown dwarf'', which is not accurately modeled by worldbuilders and therefore not included in the Madelineverse.
  \item 10\textsuperscript{25} kg is largely arbitrary, picked on the grounds that Uranus is more massive than this, while Earth isn't.
  \item 10\textsuperscript{24} kg is also largely arbitrary, tailored to the Madelineverse and its existing objects, rather than the Solar System.
  \item 10\textsuperscript{22} kg is roughly the mass of objects whose status as planets, as opposed to dwarf planets, is hotly debated, including Pluto and Eris.
  \item Below 10\textsuperscript{19} kg, objects are generally not spherical, much less rounded by the force of gravity.
\end{itemize}
\newpage

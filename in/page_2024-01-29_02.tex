\section{Standard for Celestial Fact Sheets Describing Planetary Objects}
\emph{Chronological ID:} \texttt{2024-01-29:02}

\emph{Structural ID:} \texttt{1.4.1.1}

Each planetary object (\emph{M} $>$ 10\textsuperscript{24} kg), if it comes within the focus of description, gets its own page, which includes the following data:

\begin{itemize}
  \item \textsc{All planetary objects:} mass, standard gravitational paramter, current age, radius (approximating the object as a sphere), density
  \item \textsc{Stars only:} maximum age (time expected to stay in the main sequence), luminosity, absolute magnitude, temperature of peak radiation, spectral class, position in the galaxy (mean distance from galactic core, current distance from galactic core, current distance from galactic plane), metallicity
  \item \textsc{Terrestrial planets and gas giants:} surface gravity, escape velocity, orbital parameters (semi-major axis, eccentricity, pericenter, apocenter, orbital period, inclination), sidereal rotation period, solar day length (for objects within 4 AU of their star), axial tilt, albedo (geometric and Bond)
  \item \textsc{Terrestrial planets only:} atmospheric pressure and density at surface
  \item \textsc{Currently habitable terrestrial planets only:} greenhouse effect, mean surface temperature
\end{itemize}

All data, whenever possible, is given two units: SI units, directly derived from their official definitions, and ``astronomical units'', described in relation to what object the object being described is most similar to.

The data does not adhere to a universal standard of precision; for example, a habitable terrestrial planet's orbital period might have extreme precision, required for any calendar building, while its surface gravity only matters to the precision of 1 m$\cdot$s\textsuperscript{-2}.
\newpage

\section{Standard for Celestial Fact Sheets Describing Sub-Planetary Objects}
\emph{Chronological ID:} \texttt{2024-01-30:01}

\emph{Structural ID:} \texttt{1.4.1.2}

Sub-planetary objects (\emph{M} $<$ 10\textsuperscript{24} kg) require much less data to be described, so that one page can fit 10 descriptions. Only the following are deemed essential:

\begin{itemize}
  \item \textsc{All sub-planetary objects:} mass, radius, orbital parameters (semi-major axis, eccentricity, inclination), geometric albedo
  \item \textsc{Major and middling sub-planetary objects:} Bond albedo
\end{itemize}

Unlike pages describing planetary objects, pages describing sub-planetary objects \emph{do} have strict precision requirements:

\begin{itemize}
  \item \textsc{Mass (major sub-planetary objects):} 10\textsuperscript{19} kg
  \begin{itemize}
    \item \textsc{Middling sub-planetary objects:} 10\textsuperscript{18} kg for values above 10\textsuperscript{21} kg, 10\textsuperscript{17} kg otherwise
    \item \textsc{Minor sub-planetary objects:} 10\textsuperscript{16} kg for values above 2 $\times$ 10\textsuperscript{18} kg, 10\textsuperscript{15} kg otherwise
  \end{itemize}
  \item \textsc{Radius:} 1 km for values above 1000 km, 100 m otherwise
  \item \textsc{Semi-major axis (moons):} 0.0001 LD for values under 10 LD, 0.01 LD otherwise
  \begin{itemize}
    \item \textsc{Debris disc objects:} 0.0001 AU for values under 10 AU, 0.01 AU otherwise
  \end{itemize}
  \item \textsc{Eccentricity:} 5 decimal places for values under 0.1, 3 decimal places otherwise
  \item \textsc{Inclination:} 3 decimal places for values under 10$\degree$, 2 decimal places otherwise
  \item \textsc{Albedo:} 3 decimal places
\end{itemize}
\newpage

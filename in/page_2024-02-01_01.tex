\section{Plate Tectonics Introduction}
\emph{Chronological ID:} \texttt{2024-02-01:01}

\emph{Structural ID:} \texttt{2.2.2}

Just like any planet whose internal structure's outer layers include a thin solid crust resting atop a molten mantle, the *Earth is subject to motions of the mantle material disturbing the crust material over hundreds of millions of years. This causes the crust material to crack into distinct parcels, which then move in relation to one another and constantly gain and lose ``territory''. This process is known as plate tectonics, and can be traced to the *Earth's earliest days, based on evidence such as the climate and geomagnetic record of the crust.

Many aspects of happenstances on the *Earth are, no doubt, influenced by the constant presence of the *Moon in the same place relative to the planet. However, counter-intuitively, the difference in tidal forces throughout the day is minor due to the low eccentricity of the *Moon's orbit around the *Earth. What is true over short timescales is also true over geological timescales, and tidal forces influence the *Earth's plate tectonics very little. Nevertheless, it has been observed that the sub-lunar point doesn't ``like'' being a plate boundary, and the plate underneath is more likely to rotate than to move laterally.

Reconstruction of the *Earth's plate tectonics is limited by the fact that most of the crust, especially that underneath the oceans, is less than a billion years old. However, continental crust survives for longer periods than that, and from movements of the continental crust, cycles can be observed, with the state of the continental plates going from forming one supercontinent to forming several disparate, smaller continents back to one supercontinent. Currently, the *Earth is in a supercontinent phase, with the majority of the supercontinent being situated on the side of the *Earth facing the *Moon.

[Confidential to those reading an early version of Treatise 1, a reconstruction of plate tectonics will be presented going forward in time, starting at an arbitrary ``year 0'' and moving forward 50 million *Earth years (approx. 45 million Earth years) at a time. However, once a suitable configuration is reached, all times will be retconned to ``Before Present'' (BP), counting backwards from the final configuration up to the initial one. This notice will then disappear, being replaced with the description of a major geological event that affects the *Earth.]
\newpage

\section{Does it Matter if the Madelineverse Has an Associated Story?}
\emph{Chronological ID:} \texttt{2024-02-02:01}

\emph{Structural ID:} \texttt{1.3.1}

Typically, worldbuilding projects and fictional conlangs are well-known only by extension of the associated work of fiction being well-known: most notably, the Elvish languages and the Silmarillion would not be known without Lord of the Rings. Knowing this, worldbuilders might feel pressured to write a story solely to popularise their worldbuilding project: for example, Artifexian and Biblaridion often speak about the ``book that I'll never write''.

The author comes primarily from a writing-related background, and originally saw worldbuilding as little more than a tool to give additional depth to a story. For example, one of the author's recent projects, Inside Out reImagined, originated as a way to take some basic concepts from Disney·Pixar's Inside Out and connect them in a way which would have a significant ``base'', and by extending every single in-universe concept from that base, provide worldbuilding on a level similar to J.R.R. Tolkien's.

The Madelineverse itself, likewise, only got a start because of the need to extend a story. Precursors to the Madelineverse first acknowledged in section \texttt{1.3} only had an ad-hoc ``worldbuilding'' that allowed itself to completely break its own rules. This was unsatisfactory going into the first iterations of the Madelineverse, and coupled with the author getting into worldbuilding as an art form by itself, a precursor to this treatise was started.

However, the current iteration of the Madelineverse is being developed with a new ethos, namely that the worldbuilding comes first, and only when the worldbuilding is at a sufficiently satisfactory level, a story can follow. For example, an original story (not fanfiction) requires original character names, and in order to even be able to claim to be set in the Madelineverse, the character names must be sourced from an in-universe conlang. To this extent, and since the author recognises that the current iteration of the Madelineverse may never get to a sufficiently satisfactory level, a claim that a Madelineverse story will ever arise, or that all of the worldbuilding is ``just for a story'', is never made.
\newpage

\section{Galaxy and Stellar Neighbourhood Overview}
\emph{Chronological ID:} \texttt{2024-02-08:01}

\emph{Structural ID:} \texttt{2.11}

The *Sun is located in a spiral galaxy of type Sc (henceforth the *Galaxy), at a medium distance from the core, in what's known as the ``galactic habitable zone'', and relatively close to the galactic plane. The galaxy itself is about 23,000 parsecs (75,000 light-years) in radius and is estimated to have about 7 $\times$ 10\textsuperscript{11} stars. The *Sun orbits the core of the *Galaxy in a near-circular orbit and regularly moves in and out of the spiral arms; currently, it is located outside of any spiral arm, in a region that is relatively sparse in stars.

The night sky of the *Earth reflects this, being comparatively sparse in stars, particularly bright ones; the range of apparent magnitudes of *Gas Giant 1 when seen from the *Earth, that varies between $-0.44$ and +0.01 (although the lower magnitudes are generally unseen by the local populace), contains only two stars, and the range of apparent magnitudes above that contains none.

Significant landmarks in the night sky outside the *Sun's planetary system include:

\begin{itemize}
  \item The representation of the galactic plane itself, dominated by a dark band in the middle due to the *Sun's relative closeness to the galactic plane.
  \item Three supernova remnants originating from supernovae that were visible to the naked eye in the modern era of the local civilisation. They were all of type Ia, and ordered by maximum brightness, they took place 596, 100 and 590 local years ago. Though other signs of superstition have gone away long before any of the three events, the events themselves were still believed to be ``signs from the higher forces'' that the behaviour of the local populace as a whole must change.
  \item Another spiral galaxy, about 900,000 parsecs (2.9 $\times$ 10\textsuperscript{6} light-years) away from the *Galaxy. The two are currently roughly moving in parallel with one another, but disturbance from galactic clusters nearby, as well as loss of gravitational energy, will eventually make the two spiral galaxies collide in an unknown but distant ($>$ 10\textsuperscript{10} local years) future.
\end{itemize}
\newpage

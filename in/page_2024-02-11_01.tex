\section{The *Earth's Atmosphere - Overview and Comparison to the *Moon}
\emph{Chronological ID:} \texttt{2024-02-11:01}

\emph{Structural ID:} \texttt{2.2.3}

The *Earth's atmosphere has evolved together with the planet since its first formation. Initially, the *Earth's atmosphere, much like that of its companion in space, the *Moon, was dominated by carbon dioxide, but this situation would not last.

Firstly, both the *Earth and the *Moon, due to tidal forces, exchanged large amounts of energy, eventually becoming tidally locked to one another. This resulted in volcanism on both worlds which produced nitrogen in their atmospheres, which, on the *Earth, came to largely replace the carbon dioxide. However, this was an asymmetric process, and the *Moon became tidally locked to the *Earth long before the reverse happened, meaning that only about 10\% of the *Moon's atmosphere is presently nitrogen.

Later on, lifeforms developed on the *Earth, that could use the local carbon dioxide to produce oxygen by the process of photosynthesis, and the remaining amount of carbon dioxide has grown to minimal levels, as all of it was largely replaced with oxygen. Likewise, the *Moon never became a conducive environment for lifeforms to develop beyond the unicellular stage, even taking that events of panspermia between the two were and remain common as a given.

During the present day, the *Earth's atmospheric composition is approximately:

\begin{itemize}
  \item 74.7\% nitrogen (N\textsubscript{2})
  \item 24.2\% oxygen (O\textsubscript{2})
  \item 1.1\% argon (Ar)
  \item 160 ppm carbon dioxide (CO\textsubscript{2})
  \item water vapour (H\textsubscript{2}O) (amount variable dependent on location)
  \item various other trace gases
\end{itemize}
\newpage

\section{The Order of the Structure and Presentation of Topics}
\emph{Chronological ID:} \texttt{2024-02-12:01}

\emph{Structural ID:} \texttt{1.5}

The Madelineverse is developed ``from the bottom up'': certain foundational topics are described to a certain level first before others are first approached. This applies to the notion of structural IDs, in which certain topics precede others in the final document ordered in structural order, first and foremost.

Section \texttt{1}, primarily about the Madelineverse itself, as a work of fiction and the worldbuilding document describing said work of fiction, has no strict order. The author freely chooses to describe topics relevant to Madelineverse production in whichever order, deeming the underlying structure largely irrelevant and mere trivium for others reading the document.

After section \texttt{1} come sections \texttt{2}, \texttt{3}, \texttt{4}, and \texttt{5}, each describing a broad topic, comparable to an entire branch of science, on the Earth. These are:

\begin{itemize}
  \item \texttt{2}: geo- and astrophysics
  \item \texttt{3}: local biology
  \item \texttt{4}: culture of the local sapients
  \item \texttt{5}: language of the local sapients
\end{itemize}

Note that though biology is given an important role in the structure, it is intentionally underdeveloped, with perhaps only the names, main characteristics and societal roles of certain species mentioned, and specifically, the local sapients have many unrealistic similarities to \emph{Homo sapiens sapiens}. This is partly so that readers on the Earth can continue to relate to the fictional setting.

After the sections given numbers are the sections given sequential Latin letters, such as \texttt{a}, \texttt{b}, \texttt{c}, etc. These are intended for ``miscellaneous'' knowledge, as well as illustrations that do not add anything to the worldbuilding described so far, merely illustrating something like the scale of the planetary system.
\newpage

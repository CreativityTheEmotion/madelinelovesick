\section{Cultural Introduction}
\emph{Chronological ID:} \texttt{2024-03-01:01}

\emph{Structural ID:} \texttt{4}

The culture of sapients on the *Earth, henceforth *humans, is primarily dominated by the sort of technology available to them. For example, as stated in section \texttt{2.2.1.1}, accurate tracking of the position of the *Moon allowed *humans living on the near side of the *Earth to navigate around, pretty quickly, and that, in turn, led to various cultures understanding that they're not really isolated by rivers and mountains, but rather, connected in a planet, leading to the establishment of a global trade network.

Other discoveries, inventions and realisations that made significant impact on *human history include:

\begin{itemize}
  \item The desire for language to be simplified so that it is easier to acquire, while at first primarily concerning foreign languages, led to the invention of a simplified phonology and grammar, which, when adapted to the existing languages' lexicon, created an artificial auxiliary language, Early Creole, which then took over as the primary language in all spheres of communication.
  \item Though various primitive religions and associated calendars existed, with the establishment of the global trade network and Early Creole, virtually all religions were discarded in favor of the scientific method, and only one religion's associated calendar, anglicised as ``Anno Fandomi'', survived.
  \item Through minor events, such as inconveniences caused by an inaccurate calendar and inflating prices, agreements were made to adjust the calendar to be more accurate and instate a consistent devaluation of the global currency.
  \item With industrialisation, soon, a realisation was made that cycles required for life on the *Earth, such as the carbon cycle, were being disrupted. As such, parties to the global trade network generally agreed to tone down industrial expansion; though, even one corporation's disobedience could cause the entire global trade network to disband.
\end{itemize}
\newpage

\section{The Anno Fandomi Calendar}
\emph{Chronological ID:} \texttt{2024-03-10:01}

\emph{Structural ID:} \texttt{4.1}

The values of time relevant to calendar building on the *Earth, to 10 significant figures, are as follows (not contradicting, the values in section \texttt{2.2}):

\begin{itemize}
  \item the tropical year of the *Earth is 27,188,519.21 SI seconds (or 314.6819353 days, or 0.8615521843 Julian calendar years)
  \item the synodic rotation period of the *Earth is 120,770.4662 SI seconds (or 33.54735172 hours, or 1.397806322 days)
  \item the synodic rotation period of the *Moon is exactly the same as that of the *Earth, as they are tidally locked to each other
\end{itemize}

Note that these figures are based on averages over a period of the last 2000 tropical years of the *Earth (the exact amount of time that the calendar to be discussed has been in place, counting from the calendar's epoch).

From this, the principal number defining the calendar can be derived:

\begin{itemize}
  \item one local year is 225.1255631 days
\end{itemize}

There is a variety of ways in which ancient cultures could have chosen to see this number; perhaps the most pleasing mathematically could be 15 intervals of 15 days each plus a drift of 0.1255631 days per year (about one day per 8 years). However, the system used by the dominant calendar of the *Earth in the present day, whose name is henceforth anglicised as \textsc{Anno Fandomi} (AF) (with dates predating the epoch being known as \textsc{Before the Sights} (BS)), instead admits 4 seasons of 56 days, which can then be subdivided into either 7 weeks of 8 days or 8 weeks of 7 days, plus a drift of 1.1255631 days per year (about 0.28139078 days per season, or 9 days per 32 seasons (or 8 years)).

Further details about the implementation and adoption of Anno Fandomi are forthcoming.
\newpage

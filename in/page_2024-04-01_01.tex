\section{Linguistic Introduction}
\emph{Chronological ID:} \texttt{2024-04-01:01}

\emph{Structural ID:} \texttt{5}

The *Earth, much like landmasses of comparable area and population, started out with a healthy linguistic diversity. However, around 100 BS, there was a sudden increase in focus on making education as simple as possible, on all fronts. The existence of a multitude of languages was seen as detrimental to this goal, and an anonymous scholar deveolped a phonology, syllabic writing system and set of grammatical rules, claimed not to distort the logic of anyone's native language.

As the simplified grammar was learned, people found that communication was made easier, even across the language barrier. Thus, a lexicon started taking form, and soon enough, the creolised language becaome known to the vast majority of the *Earth's population. Another few generations later, the ancestral languages that had previously dominated the *Earth went largely extinct, establishing the creolised language's dominance that continues to the setting's present day, wherein only one language is used in professional, educational, literary, and domestic affairs. Alongside trivial navigation, the emergence of a single language is credited as the greatest contributor to the emergence of a global trade network.

This is not to say that the common language of the populace didn't evolve. Stages of language evolution generally understood and documented by modern scholars include:

\begin{itemize}
  \item Early Creole (ca. year 0)
  \item Middle Language (ca. 400 AF)
  \item New Language (ca. 800 AF)
  \item Updated Spelling (ca. 1200 AF)
  \item Reformed Words (ca. 1600 AF)
  \item Modern Grammar (ca. 2000 AF)
\end{itemize}
\newpage

\section{Early Creole Phonology}
\emph{Chronological ID:} \texttt{2024-05-01:01}

\emph{Structural ID:} \texttt{5.1}

Early Creole has the following 11 consonants and 3 vowels:

\begin{tabular}{l|l|l|l|l|}
\cline{2-5}
                                  & Labial                       & Alveolar & Velar           & Glottal                      \\ \hline
\multicolumn{1}{|l|}{Nasal}       & m                            & n        &                              &                              \\ \hline
\multicolumn{1}{|l|}{Plosive}     & p                            & t        & k                            & \textglotstop \textless{}'\textgreater{} \\ \hline
\multicolumn{1}{|l|}{Fricative}   & f                            & s        & \textesh \textless{}š\textgreater{} &                              \\ \hline
\multicolumn{1}{|l|}{Approximant} & \textscriptv \textless{}v\textgreater{} & l        &                              &                              \\ \hline
\end{tabular}

\begin{tabular}{l|ll|}
\cline{2-3}
                            & \multicolumn{1}{l|}{Front} & Back \\ \hline
\multicolumn{1}{|l|}{Close} & \multicolumn{1}{l|}{i}     & u    \\ \hline
\multicolumn{1}{|l|}{Open}  & \multicolumn{2}{l|}{a}            \\ \hline
\end{tabular}

The phonotactics of Early Creole are best described at the word level, not the syllable level. The following word structures are admitted, wherein each part is mandatory:

\begin{itemize}
  \item CV (33 possible words)
  \item CVn (33 possible words)
  \item CVCV (1,089 possible words)
  \item CVCVn (1,089 possible words)
\end{itemize}

The CV and n segments are considered to be moras, while the CV and CVn segments are considered to be syllables. in mult-mora words, stress always falls on the second-to-last mora.

The writing system admits moras, not syllables, at this point. <n> is a simple horizontal line, while CV moras are written with glyphs fitting in a 3:2 box, with the top horizontal line always present and the bottom horizontal line never present. This allows mora graphemes to be stacked, forming combined word graphemes, which are read left-to-right, top-to-bottom.
\newpage
